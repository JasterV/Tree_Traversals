\documentclass{article}
\usepackage[utf8]{inputenc}
\usepackage[spanish]{babel}
\usepackage[pdftex]{graphicx}
\usepackage{listings}
\usepackage{parskip}
\usepackage{fancyhdr}
\usepackage{amsmath}
\usepackage{indentfirst}
\usepackage{vmargin}
\setmarginsrb{3 cm }{ 2.5cm }{3 cm }{2.5 cm}{1 cm}{1.5 cm}{1 cm}{1.5 cm}
\title{Binarytree}
\author{Martin, Francisco Manuel \\
Martinez, Victor
}

\makeatletter
\let\thetitle\@title
\let\theauthor\@author
\let\thedate\@date
\makeatother
\pagestyle{fancy}
\fancyhf{}
\rhead{\theauthor}
\lhead{\thetitle}
\cfoot{\thepage}
%\addto\captionspanish{
\renewcommand*\contentsname{Índex}%}
\setlength{\parindent}{1cm}

\begin{document}
    \begin{titlepage}
        \centering
        \vspace*{0.5 cm}

        \textsc{\LARGE Universidad Politécnica de Lleida}\\[2.0 cm]	% University Name
        \textsc{\large Practica 5}\\[0.5 cm]				% Course Name
        \rule{\linewidth}{0.2 mm} \\[0.4 cm]
        { \huge \bfseries \thetitle}\\
        \rule{\linewidth}{0.2 mm} \\[1.5 cm]

        \begin{minipage}{0.4\textwidth}
            \begin{flushleft} \large
            \emph{Author:}\\
            \theauthor
            \end{flushleft}
        \end{minipage}~
        \begin{minipage}{0.4\textwidth}
            \begin{flushright} \large
            \emph{DNI:} \\
            48057095k \\
            78100640T% Your Student Number
            \end{flushright}
        \end{minipage}\\[2 cm]

        {\large \thedate}\\[2 cm]

        \vfill

    \end{titlepage}
    \tableofcontents
    \pagebreak

    \section{Introducción}

    \section{Tarea 1: Implementación de los arboles binarios}
    \subsection{Constructores}
    En esta sección solo vamos a comentar un pequeño detalle sobre el tercer constructor. \newline
    Tratamos \textbf{left} y \textbf{right} de manera que, si uno de ellos es null inicializamos la instancia de Node correspondiente a null. \newline
    De esta manera evitamos un NullPointerException en caso de intentar acceder al atributo root del arbol en cuestión.

    \subsection{método equals}
    \subsubsection{equals de LinkedBinaryTree}
    En primer lugar comprobamos que el objecto de tipo \textit{Object} pasado como parámetro apunte a una instancia de LinkedBinaryTree
    Si es así, realizamos un cast y ejecutamos el método estático \textit{recEquals} de la clase Node que comprobará recursivamente si los \textit{roots} de los arboles són iguales.
    \subsubsection{equals de Node}
    Aunque este método no lo usamos en LinkedBinaryTree ya que directamente empleamos el método \textit{recEquals}, nunca está de más definir el método equals dentro de una classe.
    En este caso comprobamos que el parámetro \textit{obj} apunte a una instancia de Node y
    ejecutamos el método \textit{recEquals}
    \subsubsection{método recEquals}
    En primer lugar comprobamos si alguno de los parámetros apunta a null, si es así devolvemos el resultado de igualar los dos parámetros (si uno es null el otro también debe serlo para cumplir la igualdad).
    En caso contrario seguimos comprobando si los dos nodos són iguales (Contienen el mismo elemento y sus nodos derecho e izquierdo són también iguales).
    \newline

    \section{Tarea 2: Recorridos iterativos en árboles binarios}

    \subsection{Interfaz Traversals}
    \textbf{ ¿Qué diferencias provocaría que la interfaz fuera genérica y los
    métodos no?}\\
    En caso de que los métodos sean genéricos y la clase no, podremos usar los métodos con cualquier BinaryTree sin importar de que tipo sean los elementos que contiene.

    En cambio, si implementamos la interfaz de manera genérica y los métodos no-genericos, solo podremos usar dichos métodos con arboles binarios que contengan elementos del mismo tipo con el cual hemos inicializado la instancia de \textbf{Traversals}, \newline por lo cual necesitaremos crear una instancia de \textbf{Traversals} para cada arbol binario que definamos con un tipo distinto.

    \subsection{Clase IterativeTraversals}
    En esta clase implementamos los diferentes recorridos en profundidad sobre arboles binarios de manera iterativa. \newline
    Los stacks con los que trabajamos contendrán elementos de tipo BinaryTree, y así iremos comprobando si están vacíos o no para realizar las operaciones.\newline
    En la explicación de los siguientes métodos voy a referirme algunas veces a \textit{el último elemento introducido el en stack} como el elemento que contiene el root del arbol que es el objeto con el que realmente trabajamos.

    \subsubsection{Preorder}
    Como debemos recorrer los nodos con el patrón \textit{Parent}, \textit{left}, \textit{right}, en este algoritmo empezamos añadiendo al stack el arbol.\newline
    Vamos añadiendo a la lista el último elemento introducido en el stack y llenando el stack con los hijos derecho e izquierdo del elemento, tal y como lo plantearíamos en el algoritmo recursivo.

    \subsubsection{Inorder}
    La idea del algoritmo implementado para éste recorrido (\textit{left}, \textit{parent}, \textit{right}) se basa en posicionarse en el elemento más a la izquierda del árbol y, en cuanto lo encontramos lo añadimos a la lista, añadimos a su padre (que estamos seguros de que será el último elemento añadido en el stack) y visitamos el hijo derecho. \newline
    Seguimos hasta que el stack esté vacío y el último arbol visitado esté vacío también.

    \subsubsection{Postorder}
    El algoritmo implementado en éste recorrido (\textit{left}, \textit{right}, \textit{parent})
    se basa en una idea un poco diferente a las anteriores.\newline
    Se trata en mirar el recorrido de manera inversa, por lo que recorreremos de \textit{Parent} a \textit{right} a \textit{left} y iremos añadiendo los elementos al principio de la lista y no al final. De esta manera el código nos queda muy simple y prácticamente igual a la implementación de PreOrder solo que cambiando el orden en que añadimos los hijos.  \newline
    Cabe destacar que es necesario que la lista sea de tipo \textit{LinkedList} para que las inserciones sean óptimas, en caso de ser un \textit{ArrayList} estaríamos aumentando el coste del algoritmo prácticamente de manera exponencial.


    \section{Tarea 3: Reconstrucción del árbol binario}
    La idea en la resolución de éste método se basa en los dos argumentos siguientes:
    \newline
    El último elemento de la lista que contiene los elementos en postOrden siempre será el root del árbol.
    \newline
    Debemos encontrar la posición de ese elemento en la lista de inOrden para de esta manera poder separar los elementos que se encuentran a su izquierda y a su derecha como árbol izquierdo y árbol derecho.
    \newline
    Siguiendo estas dos premisas implementamos el método de manera recursiva mediante sublistas.

    \section{Conclusiones}
    En esta practica hemos aprendido mucho sobre el comportamiento de genéricos
    apartir de uno de los errores que cometimos en el primer apartado, en este
    apartado pretendíamos hacer un cast de \textit{object} usando los comodines
    \textbf{<E>} hasta darnos cuenta que no podríamos usarlos, todo sucedió por que
    el compilador no nos devolvía ningún error. Cambiamos los comodines a
    \textbf{<?>} para poder hacer el cast. Pero estuvimos investigando el porque que
    hacia que nuestro cast no se quejaba haciendo test exhaustivos para encontrar el
    error.

    Otro aprendizaje valioso surjio en la tarea tres cuando queriamos ver si suponia
    un sobre coste el uso de sublistas en lugar de indices porque en un principio
    creiamos que este uso haria una duplicacion de listas. Despues de repasar el
    codigo usado en las sublistas vimos que con el uso de indices como en nuestra
    idea, permitia hacer vistas a la lista original y cuando hacias una sublista de
    una sublista hacias una vista de la sublista que apuntaba a la lista original.


\end{document}